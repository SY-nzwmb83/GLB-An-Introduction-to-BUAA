\documentclass[
    12pt,
    aspectratio=169
]{beamer}

\usetheme{BUAA}
\usepackage[UTF8]{ctex}
\usepackage{tabularx}

\title{北京航空航天大学简介}
\subtitle{An Introduction to BUAA}
\author{高景澄}
\date{2025 年 4 月}
\institute{synzwmb83@outlook.com}

\begin{document}

\begin{frame}
    \titlepage
    \begin{figure}[htpb]
        \begin{center}
            \includegraphics[width=0.13\linewidth]{pic/buaa_log.eps}
        \end{center}
    \end{figure}
\end{frame}

\section{学校简介}

\begin{frame}{历史沿革}
    \begin{itemize}
        \item 始于 1952 年,由清华大学等八校航空系合并组建,为北京航空学院
        \item 1959 年,被中共中央指定为全国重点大学
        \item 1988 年,更名为北京航空航天大学
        \item 1997 年,进入国家“211 工程”重点建设高校行列
        \item 2001 年,进入国家“985 工程”重点建设高校行列
        \item 2007 年,沙河校区开工建设
        \item 2022 年,建校 70 周年,形成“一校两区”办学格局
    \end{itemize}
\end{frame}

\begin{frame}{学校概况}
    \begin{columns}[T]
        \column{0.5\textwidth}
            \begin{itemize}[<+->]
                \rmfamily
                \item 新中国首所航空航天高等学府
                \item 中国航天科技集团全面战略合作伙伴
                \item “国防七子”(“G7 联盟”)之一
                \item 隶属于工信部
                \item 校友会全国第 19,软科全国第 12,QS 世界第 452(2024 年数据)
            \end{itemize}    
        \column{0.5\textwidth}
            \begin{itemize}[<+->]
                \rmfamily
                \item 校区:学院路校区、沙河校区、杭州校区(中法航空学院)
                \item 知名校友
                    \begin{itemize}[<+->]
                        \item 邓小刚 \ttfamily 少将、空气动力学专家、国防科大原校长
                        \item 怀进鹏 \ttfamily 现任教育部部长
                    \end{itemize}
            \end{itemize}
    \end{columns}
\end{frame}

\begin{frame}{科研成果}
    \begin{itemize}[<+->]
        \item \rmfamily 中国第一架自主设计、制造的轻型旅客机 - “北京一号”
        \item \rmfamily 中国第一枚探空火箭 - “北京二号”
        \item \rmfamily 中国第一架无人驾驶飞机 - “北京五号”
        \item \rmfamily 北京系列遥感卫星
    \end{itemize} 
\end{frame}

\section{本科专业}

\begin{frame}{工科专业简介}
    北航工科分为信息类和航空航天类。
\end{frame}

\begin{frame}{航空航天类(航空航天)}
    航空航天工程学科位列全球第一‌‌(2024 软科排名)
    \begin{itemize}[<+->]
        \item   \srfamily 五系 - 航空科学与工程学院(航空)\\
                \srfamily 十五系 - 宇航学院(航天)\\
                \ttfamily 无需多言。飞行器动力、飞行器设计、飞行器控制与信息、‌智能飞行器、空气动力学等
        \item \srfamily 四系 - 能源与动力工程学院:\ttfamily 专注于航空发动机等领域,飞机发动机设计、工程热物理、推进系统优化等
        \item \srfamily 一系 - 材料科学与工程学院:\ttfamily 材料设计、制备优化、应用,尤其在航空航天轻量化材料领域优势显著
    \end{itemize}
\end{frame}

\begin{frame}{航空航天类(综合性)}
    航空航天工程学科位列全球第一‌‌(2024 软科排名)
    \begin{itemize}[<+->]
        \item \srfamily 十三系 - 交通科学与工程学院:\ttfamily 飞行器适航工程(民航适航审定、航空器安全性评估)、车辆工程
        \item \srfamily 三系 - 自动化科学与电气工程学院:\ttfamily 自动化、电气工程及其自动化、机器人工程等方向
        \item \srfamily 十七系 - 仪器科学与光电工程学院:\ttfamily 测控技术与仪器、‌光电信息工程‌等方向,注重传感技术、自动测试、光电载荷等领域
    \end{itemize}
\end{frame}


\begin{frame}{信息类}
    无需多言。
    \begin{itemize}[<+->]
    \item \srfamily 计算机学院、软件学院
    \item \srfamily 电子信息工程学院:\ttfamily 无线电系发展而来,涵盖信息与通信工程、电子科学与技术、交通运输工程、集成电路
    \end{itemize}
\end{frame}

\section{招生体系}

\begin{frame}{高考招生(招生情况)}
    \begin{block}{历年分数线与位次}
        \rmfamily 
        均为江苏本科批普通类(首选物理),其余不予展示。
        \begin{itemize}
            \item 2024年:657,位次 3026
            \item 2023年:659,位次 2949
            \item 2022年:631,位次 3004
        \end{itemize}
    \end{block}
\end{frame}

\begin{frame}{高考招生(专业分配)}
    \begin{block}{录取后专业分配}
        \rmfamily
        大类招生,分为四大类:‌航空航天类、‌信息类(‌工科试验班),理科试验班类和社会科学试验班。\\
        学生入学后先进入大类学习,经过 1-2 年通识教育后,根据兴趣、成绩及专业容量进行专业分流‌。\\
        分流主要按分数优先原则、专业调剂机制。
    \end{block}
\end{frame}
    
\begin{frame}{强基计划(报考)}
    \begin{block}{报考条件}
        \rmfamily
        \begin{itemize}
            \item 高考成绩优异的考生
            \item 相关学科领域具有突出才能和表现的考生,在对应竞赛中获国二及以上奖项
            \begin{itemize}
                \item 数学与应用数学:数学竞赛
                \item 信息与计算科学:数学或信息学竞赛
                \item 应用物理学:物理竞赛
                \item 化学:化学竞赛
                \item 工程力学:数学或物理竞赛
                \item 航空航天类:数学或物理竞赛
            \end{itemize}
        \end{itemize}
    \end{block}
\end{frame}

\begin{frame}{强基计划(测试)}
    \rmfamily
    \begin{block}{测试方式}
        校测分笔试、面试、体测 \\
        最终成绩为高考成绩 $\times$ 0.85 + 校测成绩 $\times$ 0.15
    \end{block}
\end{frame}

\section{校园生态}

\begin{frame}{学习环境}
    \rmfamily
    \begin{columns}[T]
        \column{0.5\textwidth}
            \begin{itemize}
                \item 国家实验室 12 个,年科研经费超 50 亿
                \item “冯如杯”科创大赛,年立项 2000+
                \item 本科生深造率 75\%(2024届)
            \end{itemize}
        \column{0.5\textwidth}
            \begin{itemize}
                \item 诺贝尔奖得主讲座
                \item 24 小时开放实验室 18 个
                \item 师生比 1:6,小班化教学
            \end{itemize}
    \end{columns}
\end{frame}

\section{职业发展}

\begin{frame}{就业前景与方向}
    \rmfamily 
    2024 年数据:就业率 89\%
    \begin{itemize}[<+->]
        \item 航空航天领域:中国航天科工集团公司、‌中国航空科技集团公司、‌中国航空工业集团有限公司、‌中国商飞、中国航发‌
        \item 军工方向
        \item 信息科技与互联网领域‌:华为技术有限公司、阿里巴巴‌、字节跳动、‌腾讯、百度‌
        \item 智能制造与机器人领域‌:大疆、特种机器人开发‌(军用机器狗、空间站机械臂等)
        \item 科研院所与高等教育单位
    \end{itemize}
\end{frame}

\section{The End}

\begin{frame}{The End}
    \begin{itemize}
        \rmfamily
        \item \small 本 beamer 源码开源于 Github,可下载 \\
        \item \small 遵从 CC-BY-NC 4.0 协议,允许编辑 \\
        \item \small 使用 beamer 主题为 BUAA,版权归其制作人所有 \\[0.5cm]
    \end{itemize}
    \begin{center}
        \Huge \srfamily 感谢观看\\
        \huge \rmfamily Thanks!\\[0.5cm] 
        \Large Powered By \LaTeX \\
        \large 鸣谢:Overleaf
    \end{center}
\end{frame}

\end{document}